\documentclass[11pt,a4paper]{article}

\usepackage[utf8x]{inputenc}   % omogoča uporabo slovenskih črk kodiranih v formatu UTF-8
\usepackage[slovene]{babel}    % naloži, med drugim, slovenske delilne vzorce
\usepackage{hyperref}


\title{Analiza treh primerov plagiarizma}
\author{Aljoša Rakita\\
aljo.aljo.aljo@hotmail.com\\
\ \\
Fakulteta za računalništvo in informatiko\\
Univerza v Ljubljani
\date{\today}         
}


\begin{document}
\maketitle

S pomočjo informacij, ki jih najdeš na spletu, obravnavaj tri javne primere očitkov plagiarizma. Vsaj dva primera naj bosta iz Slovenije.

\section{Prvi primer}

Ministrica Klavdija Markež je zaradi očitkov, da je njeno magistrsko delo plagiat, premierju Miru Cerarju ponudila odstop, ta pa ga je sprejel. Antiplagiatorski program je namreč ugotovil, da se njena celotna magistrska naloga v 37 odstotkih prekriva z diplomsko nekoga drugega, ki ni nikjer citirana. \cite{markez}

Bivši ministrici so naziv magistra odvzeli, vendar se ji je pod drugačnim priimkom uspelo zaposliti kot direktorica Turistično gostinske zbornice Slovenije.


\section{Drugi primer}

 Računalniška preverba podobnosti, ki so jo na pobudo časnika Večer opravili na mariborski fakulteti za računalništvo in informatiko, je pokazala, da je pravnik Ambrožič v svojem 344 strani dolgem magisteriju prepisal skoraj celo diplomsko nalogo nekdanje študentke ekonomije. Pri tem je, čeprav imata besedili identične celotne stavke in odstavke, ni navedel kot vir, za kar kazenski zakonik predvideva denarno kazen ali zaporno kazen do enega leta.\cite{ambrozic}

Poslanec Borut Ambrožič, član pozitivne slovenije, se je branil z zanikanjem in ni odstopil med preiskavo, ki jo je izvedla posebna komisija fakultete. Le ta je ugotovila krivdo in poslancu odvzela strokovni naziv.

\section{Tretji primer}

Anonimni lovci na plagiatorje so Kanglerjevo diplomo, s katero si je pridobil naziv diplomirani ekonomist, najprej preskusili z računalniškim programom turnitin.com. Izračunal je kar 70-odstotno podobnost z viri, ki so programu dostopni na spletu.
Ker računalniška analiza ni bila dovolj zgovorna, so se dela lotili še »ročno« in že pri prvih naslovih, ki jih diplomant Kangler navaja v svojem spisku literature, naleteli na ključne vire, iz katerih naj bi prepisoval. Gre za strokovne članke iz revije za lokalno samoupravo Lex localis, pod katero sta se podpisala Boštjan Brezovnik in Žan Jan Oplotnik. \cite{kangler}

Franc Kangler, župan Maribora, je odstopil, vendar verjetno ne zaradi tega saj je bil v tistem obdobju vpleten v veliko število afer. Branil se je s tem, da je svoje diplomsko delo dal v pregled trem strokovnjakom, ki naj bi ugotovili da delo ni plagiat.

\bibliographystyle{plain}
\bibliography{literatura}

\end{document}  




