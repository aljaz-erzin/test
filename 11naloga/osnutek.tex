%%%%%%%%%%%%%%%%%%%%%%%%%%%%%%%%%%%%%%%%
% datoteka diploma-vzorec.tex
%
% vzorčna datoteka za pisanje diplomskega dela v formatu LaTeX
% na UL Fakulteti za računalništvo in informatiko
%
% vkup spravil Gašper Fijavž, december 2010
% 
%
%
% verzija 12. februar 2014 (besedilo teme, seznam kratic, popravki Gašper Fijavž)
% verzija 10. marec 2014 (redakcijski popravki Zoran Bosnić)
% verzija 11. marec 2014 (redakcijski popravki Gašper Fijavž)
% verzija 15. april 2014 (pdf/a 1b compliance, not really - just claiming, Damjan Cvetan, Gašper Fijavž)
% verzija 23. april 2014 (privzeto cc licenca)
% verzija 16. september 2014 (odmiki strain od roba)
% verzija 28. oktober 2014 (odstranil vpisno številko)
% verija 5. februar 2015 (Literatura v kazalu, online literatura)
% verzija 25. september 2015 (angl. naslov v izjavi o avtorstvu)
% verzija 26. februar 2016 (UL izjava o avtorstvu)
% verzija 16. april 2016 (odstranjena izjava o avtorstvu)
% verzija 5. junij 2016 (Franc Solina dodal vrstice, ki jih je označil s svojim imenom)


\documentclass[a4paper, 12pt]{book}
%\documentclass[a4paper, 12pt, draft]{book}  Nalogo preverite tudi z opcijo draft, ki vam bo pokazala, katere vrstice so predolge!



\usepackage[utf8x]{inputenc}   % omogoča uporabo slovenskih črk kodiranih v formatu UTF-8
\usepackage[slovene,english]{babel}    % naloži, med drugim, slovenske delilne vzorce
\usepackage[pdftex]{graphicx}  % omogoča vlaganje slik različnih formatov
\usepackage{fancyhdr}          % poskrbi, na primer, za glave strani
\usepackage{amssymb}           % dodatni simboli
\usepackage{amsmath}           % eqref, npr.
%\usepackage{hyperxmp}
\usepackage[hyphens]{url}  % dodal Solina
\usepackage{comment}       % dodal Solina

\usepackage[pdftex, colorlinks=true,
						citecolor=black, filecolor=black, 
						linkcolor=black, urlcolor=black,
						pagebackref=false, 
						pdfproducer={LaTeX}, pdfcreator={LaTeX}, hidelinks]{hyperref}

\usepackage{color}       % dodal Solina
\usepackage{soul}       % dodal Solina

%%%%%%%%%%%%%%%%%%%%%%%%%%%%%%%%%%%%%%%%
%	DIPLOMA INFO
%%%%%%%%%%%%%%%%%%%%%%%%%%%%%%%%%%%%%%%%
\newcommand{\ttitle}{Prenos stila avtorja preko globokih nevronskih mrež}
\newcommand{\ttitleEn}{Style transfer trough deep neural network}
\newcommand{\tsubject}{\ttitle}
\newcommand{\tsubjectEn}{\ttitleEn}
\newcommand{\tauthor}{Aljoša Rakita}
\newcommand{\tkeywords}{prenos stila, globoke neuronske mreže, analiza}
\newcommand{\tkeywordsEn}{style transfer, deep neural network, analysis}


%%%%%%%%%%%%%%%%%%%%%%%%%%%%%%%%%%%%%%%%
%	HYPERREF SETUP
%%%%%%%%%%%%%%%%%%%%%%%%%%%%%%%%%%%%%%%%
\hypersetup{pdftitle={\ttitle}}
\hypersetup{pdfsubject=\ttitleEn}
\hypersetup{pdfauthor={\tauthor, aljo.aljo.aljo@hotmail.com}}
\hypersetup{pdfkeywords=\tkeywordsEn}


 


%%%%%%%%%%%%%%%%%%%%%%%%%%%%%%%%%%%%%%%%
% postavitev strani
%%%%%%%%%%%%%%%%%%%%%%%%%%%%%%%%%%%%%%%%  

\addtolength{\marginparwidth}{-20pt} % robovi za tisk
\addtolength{\oddsidemargin}{40pt}
\addtolength{\evensidemargin}{-40pt}

\renewcommand{\baselinestretch}{1.3} % ustrezen razmik med vrsticami
\setlength{\headheight}{15pt}        % potreben prostor na vrhu
\renewcommand{\chaptermark}[1]%
{\markboth{\MakeUppercase{\thechapter.\ #1}}{}} \renewcommand{\sectionmark}[1]%
{\markright{\MakeUppercase{\thesection.\ #1}}} \renewcommand{\headrulewidth}{0.5pt} \renewcommand{\footrulewidth}{0pt}
\fancyhf{}
\fancyhead[LE,RO]{\sl \thepage} 
%\fancyhead[LO]{\sl \rightmark} \fancyhead[RE]{\sl \leftmark}
\fancyhead[RE]{\sc \tauthor}              % dodal Solina
\fancyhead[LO]{\sc Diplomska naloga}     % dodal Solina


\newcommand{\BibTeX}{{\sc Bib}\TeX}

%%%%%%%%%%%%%%%%%%%%%%%%%%%%%%%%%%%%%%%%
% naslovi
%%%%%%%%%%%%%%%%%%%%%%%%%%%%%%%%%%%%%%%%  


\newcommand{\autfont}{\Large}
\newcommand{\titfont}{\LARGE\bf}
\newcommand{\clearemptydoublepage}{\newpage{\pagestyle{empty}\cleardoublepage}}
\setcounter{tocdepth}{1}	      % globina kazala

%%%%%%%%%%%%%%%%%%%%%%%%%%%%%%%%%%%%%%%%
% konstrukti
%%%%%%%%%%%%%%%%%%%%%%%%%%%%%%%%%%%%%%%%  
\newtheorem{izrek}{Izrek}[chapter]
\newtheorem{trditev}{Trditev}[izrek]
\newenvironment{dokaz}{\emph{Dokaz.}\ }{\hspace{\fill}{$\Box$}}

%%%%%%%%%%%%%%%%%%%%%%%%%%%%%%%%%%%%%%%%%%%%%%%%%%%%%%%%%%%%%%%%%%%%%%%%%%%%%%%
%% PDF-A
%%%%%%%%%%%%%%%%%%%%%%%%%%%%%%%%%%%%%%%%%%%%%%%%%%%%%%%%%%%%%%%%%%%%%%%%%%%%%%%


%%%%%%%%%%%%%%%%%%%%%%%%%%%%%%%%%%%%%%%% 
% define medatata
%%%%%%%%%%%%%%%%%%%%%%%%%%%%%%%%%%%%%%%% 
\def\Title{\ttitle}
\def\Author{\tauthor, aljo.aljo.aljo@hotmail.com}
\def\Subject{\ttitleEn}
\def\Keywords{\tkeywordsEn}

%%%%%%%%%%%%%%%%%%%%%%%%%%%%%%%%%%%%%%%% 
% \convertDate converts D:20080419103507+02'00' to 2008-04-19T10:35:07+02:00
%%%%%%%%%%%%%%%%%%%%%%%%%%%%%%%%%%%%%%%% 
\def\convertDate{%
    \getYear
}

{\catcode`\D=12
 \gdef\getYear D:#1#2#3#4{\edef\xYear{#1#2#3#4}\getMonth}
}
\def\getMonth#1#2{\edef\xMonth{#1#2}\getDay}
\def\getDay#1#2{\edef\xDay{#1#2}\getHour}
\def\getHour#1#2{\edef\xHour{#1#2}\getMin}
\def\getMin#1#2{\edef\xMin{#1#2}\getSec}
\def\getSec#1#2{\edef\xSec{#1#2}\getTZh}
\def\getTZh +#1#2{\edef\xTZh{#1#2}\getTZm}
\def\getTZm '#1#2'{%
    \edef\xTZm{#1#2}%
    \edef\convDate{\xYear-\xMonth-\xDay T\xHour:\xMin:\xSec+\xTZh:\xTZm}%
}

\expandafter\convertDate\pdfcreationdate 

%%%%%%%%%%%%%%%%%%%%%%%%%%%%%%%%%%%%%%%%
% get pdftex version string
%%%%%%%%%%%%%%%%%%%%%%%%%%%%%%%%%%%%%%%% 
\newcount\countA
\countA=\pdftexversion
\advance \countA by -100
\def\pdftexVersionStr{pdfTeX-1.\the\countA.\pdftexrevision}


%%%%%%%%%%%%%%%%%%%%%%%%%%%%%%%%%%%%%%%%
% XMP data
%%%%%%%%%%%%%%%%%%%%%%%%%%%%%%%%%%%%%%%%  
\usepackage{xmpincl}
\includexmp{pdfa-1b}

%%%%%%%%%%%%%%%%%%%%%%%%%%%%%%%%%%%%%%%%
% pdfInfo
%%%%%%%%%%%%%%%%%%%%%%%%%%%%%%%%%%%%%%%%  
\pdfinfo{%
    /Title    (\ttitle)
    /Author   (\tauthor, aljo.aljo.aljo@hotmail.com)
    /Subject  (\ttitleEn)
    /Keywords (\tkeywordsEn)
    /ModDate  (\pdfcreationdate)
    /Trapped  /False
}


%%%%%%%%%%%%%%%%%%%%%%%%%%%%%%%%%%%%%%%%%%%%%%%%%%%%%%%%%%%%%%%%%%%%%%%%%%%%%%%
%%%%%%%%%%%%%%%%%%%%%%%%%%%%%%%%%%%%%%%%%%%%%%%%%%%%%%%%%%%%%%%%%%%%%%%%%%%%%%%

\begin{document}
\selectlanguage{slovene}
\frontmatter
\setcounter{page}{1} %
\renewcommand{\thepage}{}       % preprecimo težave s številkami strani v kazalu
\newcommand{\sn}[1]{"`#1"'}                    % dodal Solina (slovenski narekovaji)

%%%%%%%%%%%%%%%%%%%%%%%%%%%%%%%%%%%%%%%%
%naslovnica
 \thispagestyle{empty}%
   \begin{center}
    {\large\sc Univerza v Ljubljani\\%
      Fakulteta za računalništvo in informatiko}%
    \vskip 10em%
    {\autfont \tauthor\par}%
    {\titfont \ttitle \par}%
    {\vskip 3em \textsc{DIPLOMSKO DELO\\[5mm]         % dodal Solina za ostale študijske programe
%    VISOKOŠOLSKI STROKOVNI ŠTUDIJSKI PROGRAM\\ PRVE STOPNJE\\ RAČUNALNIŠTVO IN INFORMATIKA}\par}%
    UNIVERZITETNI  ŠTUDIJSKI PROGRAM\\ PRVE STOPNJE\\ RAČUNALNIŠTVO IN INFORMATIKA}\par}%
%    INTERDISCIPLINARNI UNIVERZITETNI\\ ŠTUDIJSKI PROGRAM PRVE STOPNJE\\ RAČUNALNIŠTVO IN MATEMATIKA}\par}%
%    INTERDISCIPLINARNI UNIVERZITETNI\\ ŠTUDIJSKI PROGRAM PRVE STOPNJE\\ UPRAVNA INFORMATIKA}\par}%
%    INTERDISCIPLINARNI UNIVERZITETNI\\ ŠTUDIJSKI PROGRAM PRVE STOPNJE\\ MULTIMEDIJA}\par}%
    \vfill\null%
    {\large \textsc{Mentor}: (prof) dr.  Franc Solina\par}%
    {\vskip 2em \large Ljubljana, 2018 \par}%
\end{center}
% prazna stran
%\clearemptydoublepage      % dodal Solina (izjava o licencah itd. se izpiše na hrbtni strani naslovnice)

%%%%%%%%%%%%%%%%%%%%%%%%%%%%%%%%%%%%%%%%
%copyright stran
\thispagestyle{empty}
\vspace*{8cm}

\noindent
{\sc Copyright}. 
Rezultati diplomske naloge so intelektualna lastnina avtorja in Fakultete za računalništvo in informatiko Univerze v Ljubljani.
Za objavo in koriščenje rezultatov diplomske naloge je potrebno pisno privoljenje avtorja, Fakultete za računalništvo in informatiko ter mentorja.

\begin{center}
\mbox{}\vfill
\emph{Besedilo je oblikovano z urejevalnikom besedil \LaTeX.}
\end{center}
% prazna stran
\clearemptydoublepage

%%%%%%%%%%%%%%%%%%%%%%%%%%%%%%%%%%%%%%%%
% stran 3 med uvodnimi listi
\thispagestyle{empty}
\vspace*{4cm}

\noindent
Fakulteta za računalništvo in informatiko izdaja naslednjo nalogo:
\medskip
\begin{tabbing}
\hspace{32mm}\= \hspace{6cm} \= \kill




Tematika naloge:
\end{tabbing}
Besedilo teme diplomskega dela študent prepiše iz študijskega informacijskega sistema, kamor ga je vnesel mentor. V nekaj stavkih bo opisal, kaj pričakuje od kandidatovega diplomskega dela. Kaj so cilji, kakšne metode uporabiti, morda bo zapisal tudi ključno literaturo.
\vspace{15mm}






\vspace{2cm}

% prazna stran
\clearemptydoublepage

% zahvala
\thispagestyle{empty}\mbox{}\vfill\null\it%
\noindent
Zahvaljujem se svojim staršem, prijeteljem in mentorju.
\rm\normalfont

% prazna stran
\clearemptydoublepage

%%%%%%%%%%%%%%%%%%%%%%%%%%%%%%%%%%%%%%%%
% posvetilo, če sama zahvala ne zadošča :-)
%\thispagestyle{empty}\mbox{}{\vskip0.20\textheight}\mbox{}\hfill\begin{minipage}{0.55\textwidth}%
%Svoji dragi Alenčici.
%\normalfont\end{minipage}

% prazna stran
\clearemptydoublepage


%%%%%%%%%%%%%%%%%%%%%%%%%%%%%%%%%%%%%%%%
% kazalo
\pagestyle{empty}
\def\thepage{}% preprecimo tezave s stevilkami strani v kazalu
\tableofcontents{}


% prazna stran
\clearemptydoublepage

%%%%%%%%%%%%%%%%%%%%%%%%%%%%%%%%%%%%%%%%
% seznam kratic

\chapter*{Seznam uporabljenih kratic}  % spremenil Solina, da predolge vrstice ne gredo preko desnega roba

\begin{comment}
\begin{tabular}{l|l|l}
  {\bf kratica} & {\bf angleško} & {\bf slovensko} \\ \hline
  % after \\: \hline or \cline{col1-col2} \cline{col3-col4} ...
  {\bf NN} & neural networks & nevronske mreže \\
  {\bf DNN} & deep neural networks & globoke nevronske mreže \\
  \dots & \dots & \dots \\
\end{tabular}
\end{comment}

\noindent\begin{tabular}{p{0.1\textwidth}|p{.4\textwidth}|p{.4\textwidth}}    % po potrebi razširi prvo kolono tabele na račun drugih dveh!
  {\bf kratica} & {\bf angleško}                             & {\bf slovensko} \\ \hline
  {\bf NN} & neural networks & nevronske mreže \\
  {\bf DNN} & deep neural networks & globoke nevronske mreže \\
%  \dots & \dots & \dots \\
\end{tabular}


% prazna stran
\clearemptydoublepage

%%%%%%%%%%%%%%%%%%%%%%%%%%%%%%%%%%%%%%%%
% povzetek
\addcontentsline{toc}{chapter}{Povzetek}
\chapter*{Povzetek}

\noindent\textbf{Naslov:} \ttitle
\bigskip

\noindent\textbf{Avtor:} \tauthor
\bigskip

%\noindent\textbf{Povzetek:} 
\noindent V tej diplomski nalogi, bo najprej razloženo kaj sploh globoke nevronske mreže so in kratka razlaga njihovega delovanja. Nato bomo nekaj besed namenili temu, da bomo sploh ugotovili kako je stil slike definiran in kaj vse predstavlja. Po tem, bomo natančno razložili in analizirali delovanje specifičnih metod za prenos stila slike preko globokih nevronskih mrež. Temu bo sledila demonstracija delovanja, kjer bomo naredili 'umetne' slike s pomočjo stila znanega slovenskega impresionista. Zaključili pa bomo z anketo, ki bo preverila kako realne so producirane slike to je ali jih človek lahko loči med seboj.
\bigskip

\noindent\textbf{Ključne besede:} \tkeywords.
% prazna stran
\clearemptydoublepage

%%%%%%%%%%%%%%%%%%%%%%%%%%%%%%%%%%%%%%%%
% abstract
\selectlanguage{english}
\addcontentsline{toc}{chapter}{Abstract}
\chapter*{Abstract}

\noindent\textbf{Title:} \ttitleEn
\bigskip

\noindent\textbf{Author:} \tauthor
\bigskip

%\noindent\textbf{Abstract:} 
\noindent This diploma will first focus on explaining what deep neural networks are and a short explanation of how do they work. Then we will say a few words about style of the image; how it is defined and what it represents. After that we will explain in detail and analyze specific methods for image style transfer via deep neural networks. That will be followed by demonstration where we'll create "fake" pictures using style of one of the slovene impressionists.
\bigskip

\noindent\textbf{Keywords:} \tkeywordsEn.
\selectlanguage{slovene}
% prazna stran
\clearemptydoublepage

%%%%%%%%%%%%%%%%%%%%%%%%%%%%%%%%%%%%%%%%
\mainmatter
\setcounter{page}{1}
\pagestyle{fancy}

\chapter{Uvod}

Za to temo sem se odločil, ker se mi zdi področje umetne inteigence zelo zanimivo in perspektivno. Naloga bo razložila in analizirala sistem za prenos stila slike preko globokih nevronski mrež. Poleg tega bo služila tudi kot neka kritična ocena, kako sposobne so sodobne metode umetne inteligence ustvariti nekaj zelo človeškega: umetnost. 


V \ref{dnn}. poglavju bomo na kratko predstavili kaj sploh globoke nevronske mreže so in kako delujejo. Videli bomo, da so le te zelo močno orodje umetne inteligence in zakaj so primerne za predstavljen problem.
Kako je stil slike definiran in kaj sploh to je, bomo razložili v \ref{stil}. poglavju, ki bo služilo tudi kot umetniška podlaga naloge.
V \ref{metode}. poglavju bomo predstavili aktualne tehnike, ki predstavljen problem rešujejo s pomočjo globokih nevronskih mrež, njihove prednosti ter slabosti in primernost uporabe. Realizacija teh metod, prikaz delovanja in izdelava 'fake' slike bo opisano v \ref{demonstracija}. poglavju. Poglavje \ref{analiza} se bo osredotočilo na analizo in interpretacijo rezultatov s pomočjo ankete.
V zaključku(poglavje \ref{zakljucek}) bo podana kritična ocena metode in končni sklep: ali je umetna inteligenca sposobna proizvesti umetniško sliko, za katero človek ni sposoben določiti ali je človeško oziroma delo računalnika.


Rezultati bodo koristili laikom, za lažje razumevanje, kot tudi strokovnjakom za pomoč pri izbiri ustrezne metode.



//TODO: razširi

\chapter{Globoke nevronske mreže}
\label{dnn}

\section{Kratka predstavitev}

Pod pojem nevronske mreže smatramo množico algoritmov, ki so ustvarjeni za zaznavo vzorcev v podatkih. Spadajo pod področje umetne inteligence, tako pa jih imenujemo zato, ker delujejo približno podobno kot človeški možgani. Začetki segajo v štirideseta leta 20. stoletja, čeprav so bili zametki NN lahko opaženi že v začetku 19. stoletja.\cite{schmidhuber2015deep}


Zgrajene so iz veliko vozlišč, kjer se dogaja dejansko računanje. V vozlišče vodi vhod, ki je ponavadi razdeljen na več delov(vektor) in vpliva na odziv vozlišča(od tu izhaja podobnost s človeškim nevronom). Le to združi vhodni vektor z množico koeficientov, katere imenujemo uteži. Te bodisi povečajo ali zmanjšajo vrednosti vhoda in tako dodeljujejo prioriteto vhodnim vektorjem za nalogo, ki se jo algoritem uči. Vrednosti vektorja, pomnožene z utežmi se seštejejo in prenesejo na izhod.


Globoke nevronske mreže je oznaka za mreže, ki so sestavljene iz več plasti nevronskih mrež. Izhod ene je vhod v drugo in tako naprej, med njimi pa ponavadi najdemo tako imenovane aktivacijske funkcije, ki določijo ali in koliko signala se prenese v naslednjo.
Pri globokem učenju se vsaka plast vozlišč uči na množici lastnosti, ki temeljijo na prejšnji plasti. Globje v nevronsko mrežo gremo bolj kompleksne lastnosti so mreže sposobne prepoznati saj se le te seštevajo in kombinirajo z lastnostmi prejšnjih plasti, temu pravimo tudi hierarhija lastnosti. \cite{schmidhuber2015deep}




\section{Uporaba}

Globoke nevronske mreže se uporabljajo za reševanje raznoraznih problemov; lahko klasificirajo ali pa gručijo. Trenutno je njihov vplih najbolj opazen pri:
\begin{itemize}
  \item Avtomatski prepoznavi govora
  \item Prepoznavi slik/objektov na sliki
  \item Procesiranju vizualne umetnosti
  \item Procesiranju naravnega jezika
  \item Toksikologiji
  \item Organiziranju odnosa s strankami
  \item Priporočilnih sistemih
  \item Bioinformatiki
  \item itd.


\end{itemize}

//TODO: razširi

\chapter{Stil slike}
\label{stil}

\section{Definicija}

V umetnosti se pojem stil slike definira kot lastnost ki nam pomaga razvrščati umetniška dela v raznolike kategorije oziroma katerikoli prepoznavni način ali metoda, ki se uporablja za izdelavo oziroma izvedbo dela. Nanaša se na vizualni izgled umetniškega dela, ki je prisoten v vseh delih istega avtorja ali nekoga iz istega obdobja, urjenja, lokacije, 'šole', umetniškega gibanja oziroma kulture.\cite{wikiStil}


Stil običajno delimo glede na časovno obdobje, državo, kulturno skupino, skupino umetnikov oziroma umetniško gibanje ter stil posameznika znotraj teh skupin.
V tej diplomski nalogi se bomo srečali s slovenskim impresionizmom, katerga bomo uporabili za učenje globokih nevronskih mrež in za izdelavo 'fake' slik.

%\cite{karayev2013recognizing}


\section{Impresionizem}


Impresionizem je umetniški stil iz druge polovice 19. stoletja, za katerega je značilno, da so umetniki dajali poudarek na natančen prikaz svetlobe, na slikah pa so bile majhne, tanke a vendar vidne poteze čopiča. Motivi so blili vsakdanji, 'običajni' dogodki z vključevanjem gibanja kot klučni elementc človeškega zaznavanja in občutenja.
Začet je bil v Parizu strani skupine umetnikov, ki so bili opaženi preko samostojnih razstav. Ime pa je dobil po sliki Clauda Moneta: Impression, soleil levant(Impresija, vzhajajoče sonce).


Na slovenskem je impresionizem prvič po baroku razgibal umetnostno življenje. Slovenski impresionisti predstavljajo prvo resnično moderno slikarsko generacijo pri nas(Ivan Grohar, Matija Jama, Rihard Jakopič, Matej Sternen).

//TODO: razširi


\chapter{Metodologija}
\label{metode}

\section{Image style transfer using convolutional neural networks}
 Predstavi sistem, ki temelji na globokih nevronskih mrežah, ki ustvari visoko-kvalitetne umetniške slike. Članek naredi korak naprej v razumevanju kako ljudje ustvarjamo in zaznavamo umetnost.\cite{gatys2016image}


\section{Preserving color in neural artistic style transfer} Nadaljevanje zgornjega članka, ki se loti problemov in pomankljivosti opisanega algoritma. Opiše preproste linearne metode, ki problem rešijo.\cite{gatys2016preserving}


\section{Texture Networks: Feed-forward Synthesis of Textures and Stylized Images}
V tem članku je predstavljen sistem, ki tudi temelji na globokih nevronskih mrežah in ustvari sliko, vendar to naredi veliko hitreje saj je večina obdelave prestavljena v fazo učenja.\cite{ulyanov2016texture}


\section{PyCharm razvojno okolje}
Za izdelavo in priredbo programskega dela naloge smo uporabili razvojno okolje PyCharm. To orodje olajša testiranje in razhroščevanje kode.


//TODO: razširi


\chapter{Demonstracija delovanja}
\label{demonstracija}

//TODO: implementacija metode in izelava slik(e)


\chapter{Analiza in interpretacija rezultatov}
\label{analiza}

//TODO: izdelava ankete, analiza in interpretacija rezultatov


\chapter{Zaključek}
\label{zakljucek}

//TODO: zaključek


\newpage %dodaj po potrebi, da bo številka strani za Literaturo v Kazalu pravilna!
\ \\
\clearpage
\addcontentsline{toc}{chapter}{Literatura}
\bibliographystyle{plain}
\bibliography{literatura}


\end{document}

