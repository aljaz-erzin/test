\documentclass[11pt,a4paper]{article}

\usepackage[utf8x]{inputenc}   % omogoča uporabo slovenskih črk kodiranih v formatu UTF-8
\usepackage[slovene]{babel}    % naloži, med drugim, slovenske delilne vzorce



\title{Perspektiva}
\author{Aljoša Rakita\\
aljo.aljo.aljo@hotmail.com\\
\ \\
možni MENTOR: prof. dr. Franc Solina \\
Fakulteta za računalništvo in informatiko Univerze v Ljubljani
\date{\today}         
}


\begin{document}
\maketitle

Do konca predavanj napiši dokument na najmanj dveh polnih straneh o možni diplomski temi, v seznamu literature naj bodo trije članki iz znanstvenih revij
(tip literature v Bib\TeX u: article), kjer je vsaj eden od avtorjev pedagog na FRI. 
Za generiranje seznama literature uporabite Bib\TeX . 



\section{Motiv za diplomsko nalogo}

Zakaj me ta tema zanima? 
V tem, zadnjem letniku faksa sem ses spoznal s področjem računalniškega vida. To me je hitro pritegnilo zato sem se odločil tudi za diplomsko nalogo iz te smeri.
Zadnjih nekaj let je pri študentih arhitekture in likovnih smeri, ki so povezane z risanjem prostora po načelih prostorskih ključev oziroma linearne
perspektive, opaziti težave pri transferju realnega v likovni prostor.


\section{Kako so se s to temo dosedaj ukvarjali učitelji na FRI?}


S računalniškim vidom se je na fakulteti za računalništvo in informatiko ukvarjalo veliko profesorjev in asistentov (Danijel Skočaj, Matej Kristan, Franc Solina,Luka Čehovin Zajc ...), ki so raziskovali hierarhične modele v vizualnem sledenju \cite{hierarhicni} in bili pri tem uspešni saj predstavljena eksperimentalna analiza na obstoječih primerjalnih zbirkah podatkov pokaže, da opisana sledilnika spadata v sam vrh raziskav na področju kratkoročnega vizualnega sledenja ter se še posebej odlikujeta v sledenju netogih objektov.
Zaposleni na našem faksu so raziskovali in razvijali sisteme za avtomatsko analizo aktivnosti ekip v igri košarke \cite{pervse2009trajectory} , pri čemer so razvili sistem in njegovo učinkovitost ter robustnost predstavili v dveh igrah ter 71 primerih napada.


\section{Kaj je konkretni cilj diplomske naloge in kateri so glavni koraki do tega cilja?}

Cilj je stvaritev vmesnika ki bi ob primerni uporabi v učnem procesu, opo-
zarjal na razlike med doživljanjem sveta prek vmesnikov in brez njih ter jim
tako omogočil lažje prehajanje iz realnega (skozi digitalni) v likovni prostor.
To bomo dosegli z razvojem android aplikacije za mobilne platforme, ki je bo preprost za uporabo.


\section{S kakšnimi orodji boš prišel do cilja}

Pri izdelavi diplomske naloge bomo večino dela naredili v razvojnem okolju
Android studio, ki poenostavi razvoj mobilnih aplikacij. Z različnimi implementacijami algoritmov za zaznavanje "ključnih točk" in prepoznavanje objetov so se seveda že ukvarjali drugi ljudje. Eksperimenti z implementacijo  RANSAC, SIFT in SURF algoritmov v okolju android so pokazali, da je mogoče izvajati sledenje objektom s procesorsko močjo mobilnih naprav v skoraj realnem času \cite{olsson2009distributed}


\section{Kako bomo preizkusili rešitev ali ustreza zadanim ciljem?}

Rešitev bomo predstavili tako, da bomo demonstrirali njeno delovanje na
realnem primeru v realnem času.


\section{Zaključek: zakaj je izbrani mentor primeren za predlagano temo?}

Profesor doktor Franc Solina je predstojnk laboratorija za računlniški vid, ki se ukvarja z raziskavami na istem področju, kot je tema te diplomske naloge. Poleg tega je bil profesor mentor že velikemu številu študentov.

\bibliographystyle{plain}
\bibliography{literatura}

\end{document}  




