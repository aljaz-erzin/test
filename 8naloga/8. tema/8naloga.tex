\documentclass[11pt,a4paper]{article}

\usepackage[utf8x]{inputenc}   % omogoča uporabo slovenskih črk kodiranih v formatu UTF-8
\usepackage[slovene]{babel}    % naloži, med drugim, slovenske delilne vzorce

\usepackage[hyphens]{url}
\usepackage{hyperref}

\usepackage{graphicx}

\title{Prenos stila avtorja preko globokih nevronskih mrež}
\author{Aljoša Rakita\\
aljo.aljo.aljo@hotmail.com\\
\ \\
predvideni MENTOR: (viš. pred./doc./prof.) dr. Franc Solina \\
Fakulteta za računalništvo in informatiko Univerze v Ljubljani
\date{\today}         
}



\begin{document}
\maketitle



\section{Opis raziskovalne metode, ki je primerna za predlagano diplomsko temo}

\begin{itemize}
\item Comparative study / primerjalna študija: ta metoda se mi zdi primerna, saj je cilj moje diplomske naloge razložiti kako algoritem deluje in kakšne so moderne izvedbe le tega.

\item Simulation / simulacija: s pomočjo te metode bom simuliral delovanje in preveril učinkovitost preko spletne ankete.
\end{itemize}



\section{Seznam literature}

\begin{itemize}
\item Deep residual learning for image recognition \cite{deepImgRec}  (najprej teoretični uvod, nato simulacija in primerjava ter zaključek)
\item How transferable are features in deep neural networks? \cite{deepFeaturesTransfer} (najprej teoretični uvod, nato simulacija in primerjava ter zaključek)

\item Feature Learning in Deep Neural Networks – Studies on Speech Recognition Tasks \cite{deepFeatureSpeech} (najprej teoretični uvod, nato primerjava in zaključek)
\end{itemize}


\bibliographystyle{plain}
\bibliography{literatura}

\end{document}  




