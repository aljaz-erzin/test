\documentclass[11pt,a4paper]{article}

\usepackage[utf8x]{inputenc}   % omogoča uporabo slovenskih črk kodiranih v formatu UTF-8
\usepackage[slovene]{babel}    % naloži, med drugim, slovenske delilne vzorce

\usepackage[hyphens]{url}
\usepackage{hyperref}


\title{Prenos stila avtorja preko globokih nevronskih mrež}
\author{Aljoša Rakita\\
aljo.aljo.aljo@hotmail.com\\
\ \\
predvideni MENTOR: (viš. pred./doc./prof.) dr. Franc Solina \\
Fakulteta za računalništvo in informatiko Univerze v Ljubljani
\date{\today}         
}



\begin{document}
\maketitle

\section{Najbolj relevantna publikacija mojega predvidenega mentorja v zvezi z mojo predvideno diplomsko nalogo}

Human skin color clustering for face detection \cite{kovac2003human}, to delo predstavi zaznavo barve in gručenje, dva modela umetne inteligence, s katero se bom ubadal pri mojem diplomskem delu.


\section{Katere so tri najbolj citirane publikacije mojega predvidenega mentorja}

\subsection{V sistemu COBISS oziroma SICRIS in v Google učenjaku}


\begin{itemize}
  \item Recovery of parametric models from range images: The case for superquadrics with global deformation\cite{solina1990recovery}
  \item Human skin color clustering for face detection \cite{kovac2003human}
\item Segmentation and recovery of superquadrics \cite{jaklic2013segmentation}
\end{itemize}

\section{Kakšen je h-indeks mojega predvidenega mentorja}

H-indeks Franca Soline je nasploh 23, od leta 2013 pa 12.


\section{Strani mentorja na Google učenjaku, na Research gate in v  Academiji}

Publikacije Franca Soline lahko najdemo na naslednji akademskih portalih:
\begin{description}
\item[Google učenjak:] \url{http://scholar.google.si/citations?user=ShKuoiUAAAAJ}

\item[Research gate:] \url{https://www.researchgate.net/profile/Franc_Solina/}

\item[Academia:] \url{http://uni-lj.academia.edu/FrancSolina}

\end{description}

\bibliographystyle{plain}
\bibliography{literatura}

\end{document}  




